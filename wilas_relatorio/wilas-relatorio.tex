% !TeX root = RJwrapper.tex
\title{Relatório Econômica-Social da População da Cidade de Aracaju}
\author{by José Wilas Alves de Farias}

\maketitle


\hypertarget{introduuxe7uxe3o}{%
\section{Introdução}\label{introduuxe7uxe3o}}

Este trabalho é desenvolvido apenas como uma forma de pensar o município
de forma crítica e análitica de forma não detalhada, mas observando
pontos importantes da situação econômica e social da capital do estado
de Sergipe.

O trabalho está dividido em 5 capítulos além dessa introdução. O segundo
capítulo analisa a População Total do Município por Gênero e a Renda
Média. O terceito capítulo trata da População na Força de Trabalho,
Ocupadas e Desocupadas, Fora da Força de Trabalho e Desalentados. O
quarto capítulo estuda a População NEM-NEM por Nível de Instrução. O
quinto capítulo estuda os Jovens Desocupados por Nível de Instrução. E,
o sexto capítulo analisa as Pessoas Ocupadas por Setor Produtivo.

\hypertarget{total-da-populauxe7uxe3o-de-aracaju-por-sexo} da população da capital do estado de Sergipe é formada por
mulheres. Há crescimento apróximado entre os trimestres e de forma
crescente da população de Aracaju, com redução do quantitativo de homens
no 2º trimestre de 2022 e 1º trimestre de 2023.

\begin{longtable}[]{@{}ccccr@{}}
\caption{\label{tab1}Total da População por Sexo em Aracaju e Renda
Média, 2022.1 a 2023.1}\tabularnewline
\toprule\noalign{}
Trimestre & Homem & Mulher & Total & Renda Media \\
\midrule\noalign{}
\endfirsthead
\toprule\noalign{}
Trimestre & Homem & Mulher & Total & Renda Media \\
\midrule\noalign{}
\endhead
\bottomrule\noalign{}
\endlastfoot
2022.1 & 324.816 & 352.435 & 677.251 & 3.108,09 \\
2022.2 & 319.497 & 359.615 & 679.112 & 3.136,40 \\
2022.3 & 321.940 & 359.022 & 680.962 & 3.084,29 \\
2022.4 & 325.904 & 356.896 & 682.800 & 3.521,83 \\
2023.1 & 323.962 & 360.664 & 684.626 & 3.865,34 \\
\end{longtable}

\begin{tablenotes}
Table notes environment without optional leadin.
\end{tablenotes}

\textbf{Fonte:} IBGE \citep{ibge2023}

Já no caso das mulheres ocorreu redução no 3º e de forma mais relevante
no 4º trimestre de 2022. Olhando para o lado da renda, tem-se que
ocorreu redução no 3º trimestre de 2022, mas que retornou a sua
tragetória de crescimento no trimestre seguinte, chegando a
\texttt{13,3\%} de diferênça do 4º e o 1º trimestre de 2022 e aumento de
\texttt{24,4\%} entre o primerio trimestre de 2022 até o primeiro
trimestre de 2023, com renda média de R\$ 3.865,34.

\hypertarget{populauxe7uxe3o-na-foruxe7a-de-trabalho-fora-da-foruxe7a-de-trabalho-ocupada-desocupada-e-em-desalento-em-aracaju}{%
\section{População na Força de Trabalho, Fora da Força de Trabalho,
Ocupada, Desocupada e em Desalento em
Aracaju}\label{populauxe7uxe3o-na-foruxe7a-de-trabalho-fora-da-foruxe7a-de-trabalho-ocupada-desocupada-e-em-desalento-em-aracaju}}

De início a Tabela \ref{tab2} mostra que há cresciemtno da população da
capital do estado, mas em comparação com a Tabela \ref{tab1} demonstra
que esse aumento representa pessoas em idade não ativa. Destaca-se a
grande queda das Pessoas na Força do Trabalho, coupadas e desocupadas,
em comparação das Pessoas Fora da força de trabalho.

\begin{longtable}[]{@{}
  >{\centering\arraybackslash}p{(\columnwidth - 10\tabcolsep) * \real{0.0803}}
  >{\centering\arraybackslash}p{(\columnwidth - 10\tabcolsep) * \real{0.2190}}
  >{\centering\arraybackslash}p{(\columnwidth - 10\tabcolsep) * \real{0.2555}}
  >{\centering\arraybackslash}p{(\columnwidth - 10\tabcolsep) * \real{0.1314}}
  >{\centering\arraybackslash}p{(\columnwidth - 10\tabcolsep) * \real{0.1533}}
  >{\centering\arraybackslash}p{(\columnwidth - 10\tabcolsep) * \real{0.1606}}@{}}
\caption{\label{tab2}População na Força de Trabalho, Fora da Força de
Trabalho, Ocupada, Desocupada e em Desalento em Aracaju, 2022.1 a
2023.1}\tabularnewline
\toprule\noalign{}
\begin{minipage}[b]{\linewidth}\centering
Trimestre
\end{minipage} & \begin{minipage}[b]{\linewidth}\centering
Pessoas na força de trabalho
\end{minipage} & \begin{minipage}[b]{\linewidth}\centering
Pessoas fora da força de trabalho
\end{minipage} & \begin{minipage}[b]{\linewidth}\centering
Pessoas ocupadas
\end{minipage} & \begin{minipage}[b]{\linewidth}\centering
Pessoas desocupadas
\end{minipage} & \begin{minipage}[b]{\linewidth}\centering
Pessoas desalentadas
\end{minipage} \\
\midrule\noalign{}
\endfirsthead
\toprule\noalign{}
\begin{minipage}[b]{\linewidth}\centering
Trimestre
\end{minipage} & \begin{minipage}[b]{\linewidth}\centering
Pessoas na força de trabalho
\end{minipage} & \begin{minipage}[b]{\linewidth}\centering
Pessoas fora da força de trabalho
\end{minipage} & \begin{minipage}[b]{\linewidth}\centering
Pessoas ocupadas
\end{minipage} & \begin{minipage}[b]{\linewidth}\centering
Pessoas desocupadas
\end{minipage} & \begin{minipage}[b]{\linewidth}\centering
Pessoas desalentadas
\end{minipage} \\
\midrule\noalign{}
\endhead
\bottomrule\noalign{}
\endlastfoot
2022.1 & 367.705 & 195.409 & 314.462 & 53.243 & 11.994 \\
2022.2 & 362.663 & 197.780 & 314.610 & 48.052 & 14.807 \\
2022.3 & 356.474 & 201.020 & 309.245 & 47.229 & 15.927 \\
2022.4 & 356.726 & 198.548 & 310.155 & 46.571 & 10.245 \\
2023.1 & 344.165 & 218.839 & 301.091 & 43.073 & 11.671 \\
\end{longtable}

Para o primeiro grande grupo, PEA, tem-se que o impacto da redução
recaiu em grande parte no número de pessoas ocupadas, que saiu de
314.462 pessoas ocupadas para 301.091 pessoas, 1º trimestre de 2022 e
2023, respectivamente. Por outro lado o número de desempregados caiu de
53.243 para 43.073 pessoas, uma queda de aproximadamente \emph{10.170}
pessoas, 1º trimestre de 2022 a 2023 respectivamente, queda inferior as
de aproximadamente 13 Mil pessoas das pessoas ocupadas. É importante
salientar que a há uma tragetória de queda do número de pessoas
desocupados, mas se apresenta não de forma consistênte e que é de
extrema importância dadas que a taxa de desemprego ainda está em 13\% no
4º trimestre de 2022 e no 1º Trimestre de 2023 está em 12,5\%. A redução
apresentada no número de pessoas desocupadas em termos absolutos, não
representou melho no quadro empregatício do município, o que pode supor
é que essa massa de desempregados transferiu-se para o núero de Pessoas
Fora da Força de Trabalho e uma pequeno número foi para os desalentados,
entendimento corroborado pelo aumento desses dois indicadores.

Olhando para as pessoas em desalento, as quais representam uma parte as
PÑEA, que chegou ao auge nos 2º e 3º trimestre de 2022, chegando as
15.927 pessoas que devido a não conseguirem ocupação por um longo
período de tempo pararam de procurar emprego. Apesar da redução do
número de desalentados, ainda não se mostraram efetivas e eficientes as
medidas tomadas para mitigar essa triste realidade que está a si
transformar em ``patologia'' na capital do estado.

\begin{longtable}[]{@{}ccc@{}}
\caption{\label{tab3}Pessoas que Estudam em Aracaju, 2022.1 a
2023.1}\tabularnewline
\toprule\noalign{}
Trimestres & Estudam & Não Estudam \\
\midrule\noalign{}
\endfirsthead
\toprule\noalign{}
Trimestres & Estudam & Não Estudam \\
\midrule\noalign{}
\endhead
\bottomrule\noalign{}
\endlastfoot
2022.1 & 181.670 & 453.715 \\
2022.2 & 180.855 & 460.077 \\
2022.3 & 177.661 & 465.271 \\
2022.4 & 187.619 & 451.997 \\
2023.1 & 176.399 & 459.864 \\
\end{longtable}

Na Tabela \ref{tab3} há uma tendência de aumento do número de pessoas
que não estudam em Aracaju, apesar do aumento/retomada das pessoas a
estudar no 4º trimestre de 2022, mas já no 1º trimestre de 2023 há o
aumento do número de pessoas que não estudam em torno \emph{7.867}
(\texttt{1,7\%}) do total 459.864 pessoas (\texttt{72\%} da população
total).

\begin{longtable}[]{@{}
  >{\centering\arraybackslash}p{(\columnwidth - 8\tabcolsep) * \real{0.1134}}
  >{\centering\arraybackslash}p{(\columnwidth - 8\tabcolsep) * \real{0.1856}}
  >{\centering\arraybackslash}p{(\columnwidth - 8\tabcolsep) * \real{0.2268}}
  >{\centering\arraybackslash}p{(\columnwidth - 8\tabcolsep) * \real{0.2165}}
  >{\centering\arraybackslash}p{(\columnwidth - 8\tabcolsep) * \real{0.2577}}@{}}
\caption{\label{tab4}Pessoas Ocupadas e Desocupadas que Estudam e Não
Estudam em Aracaju, 2022.4 a 2023.1}\tabularnewline
\toprule\noalign{}
\begin{minipage}[b]{\linewidth}\centering
Trimestre
\end{minipage} & \begin{minipage}[b]{\linewidth}\centering
Ocupadas Estudam
\end{minipage} & \begin{minipage}[b]{\linewidth}\centering
Ocupadas Não Estudam
\end{minipage} & \begin{minipage}[b]{\linewidth}\centering
Desocupadas Estudam
\end{minipage} & \begin{minipage}[b]{\linewidth}\centering
Desocupadas Não Estudam
\end{minipage} \\
\midrule\noalign{}
\endfirsthead
\toprule\noalign{}
\begin{minipage}[b]{\linewidth}\centering
Trimestre
\end{minipage} & \begin{minipage}[b]{\linewidth}\centering
Ocupadas Estudam
\end{minipage} & \begin{minipage}[b]{\linewidth}\centering
Ocupadas Não Estudam
\end{minipage} & \begin{minipage}[b]{\linewidth}\centering
Desocupadas Estudam
\end{minipage} & \begin{minipage}[b]{\linewidth}\centering
Desocupadas Não Estudam
\end{minipage} \\
\midrule\noalign{}
\endhead
\bottomrule\noalign{}
\endlastfoot
2022.1 & 39.251 & 275.210 & 12.621 & 40.623 \\
2022.2 & 37.896 & 276.715 & 13.696 & 34.356 \\
2022.3 & 35.399 & 273.846 & 9.964 & 37.265 \\
2022.4 & 40.963 & 269.193 & 12.670 & 33.901 \\
2023.1 & 38.408 & 262.683 & 8.766 & 34.307 \\
\end{longtable}

Na Tabela \ref{tab4}, tem-se que algumas pessoas entender como
possitivo, mas que de fato é algo preocupante. Em que no último
trimestre de 2022 houve aumento relevante do númnero de pessoas
desempregadas e que estudavam, mas no primeiro trimestre de 2023 esse
número voltou a cair, tendo uma leve sensação que melhoras. Entretanto,
é de suma importância salientar que essa redução não refletiu no aumento
do número de empregado, mas de aumento do número de pessoas que além de
estarem fora do mercado de trabalho pioram sua situação ao deixarem de
estudar e adquirirem novas competéncias. \ldots{}

\begin{longtable}[]{@{}lcc@{}}
\caption{\label{tab5}População por Nivel de Instrução em Aracaju, 2022.4
a 2023.1}\tabularnewline
\toprule\noalign{}
Nivel & 2022.4 & 2023.1 \\
\midrule\noalign{}
\endfirsthead
\toprule\noalign{}
Nivel & 2022.4 & 2023.1 \\
\midrule\noalign{}
\endhead
\bottomrule\noalign{}
\endlastfoot
Sem instrução e menos de 1 ano de estudo & 31.559 & 29.817 \\
Fundamental incompleto ou equivalente & 179.191 & 163.326 \\
Fundamental completo ou equivalente & 31.054 & 46.958 \\
Médio incompleto ou equivalente & 40.117 & 42.550 \\
Médio completo ou equivalente & 166.959 & 165.382 \\
Superior incompleto ou equivalente & 46.322 & 45.517 \\
Superior completo & 144.415 & 142.712 \\
\end{longtable}

Como mostra a Tabela \ref{tab5}, População por Nivel de Instrução, há
redução do número de pessoas sem o curso fundamental completo, mas ainda
é preocupante essa situação levando em consideração que no 1º Trimestre
de 2023 cerca de \texttt{30\%} das pessoas ainda não concluíram o ensino
fundamental. Nesse sentido, mostra que houve redução de pessoas com
curso médio completo e acima, e podemos entender que há fuga de mentes
na capital do estado.

Na Tabela \ref{tab6}, Nivel de Instrução das Pessoas que Não Estudam,
foi constatado que no 4º Trimestre de 2022 cerca de \texttt{71\%} da
população por nível de Instrução já não estudam mais e no 1º Trimestre
de 2023 ocorre aumento desse percentual para \texttt{72\%}
(\emph{459.864} pessoas). Há uma constância n opercentual de pessoas sem
o curso fundamental completo, mas ocorreu aumento do número de pessoas
que não concluíram o ensino médio em Aracaju.

\begin{longtable}[]{@{}lcc@{}}
\caption{\label{tab6}Nivel de Instrução das Pessoas que Não Estudam em
Aracaju, 2022.4 a 2023.1}\tabularnewline
\toprule\noalign{}
Nivel & 2022.4 & 2023.1 \\
\midrule\noalign{}
\endfirsthead
\toprule\noalign{}
Nivel & 2022.4 & 2023.1 \\
\midrule\noalign{}
\endhead
\bottomrule\noalign{}
\endlastfoot
Sem instrução e menos de 1 ano de estudo & 11.798 & 12.974 \\
Fundamental incompleto ou equivalente & 94.121 & 92.719 \\
Fundamental completo ou equivalente & 22.090 & 31.777 \\
Médio incompleto ou equivalente & 26.680 & 26.442 \\
Médio completo ou equivalente & 154.250 & 156.897 \\
Superior incompleto ou equivalente & 19.829 & 17.383 \\
Superior completo & 123.229 & 121.672 \\
\end{longtable}

Logo, constata a fugam de mentes ao observar redução d número de pessoas
por escolaridade acima de ensino médio, sendo cooroborado pela redução
nesssa faixas das pessoas que deixaram de estudar.

\hypertarget{populauxe7uxe3o-que-nuxe3o-estuda-e-nuxe3o-trabalha-nem-nem-por-nivel-de-instruuxe7uxe3o-em-aracaju}{%
\section{População que Não Estuda e Não Trabalha (Nem-Nem) por Nivel de
Instrução em
Aracaju}\label{populauxe7uxe3o-que-nuxe3o-estuda-e-nuxe3o-trabalha-nem-nem-por-nivel-de-instruuxe7uxe3o-em-aracaju}}

\begin{longtable}[]{@{}lcc@{}}
\caption{\label{tab7}Nem Estudam e Nem Trabalham por Nivel de Instrução
em Aracaju, 2022.4 a 2023.1}\tabularnewline
\toprule\noalign{}
Nivel & 2022.4 & 2023.1 \\
\midrule\noalign{}
\endfirsthead
\toprule\noalign{}
Nivel & 2022.4 & 2023.1 \\
\midrule\noalign{}
\endhead
\bottomrule\noalign{}
\endlastfoot
Sem instrução e menos de 1 ano de estudo & 757 & 1.080 \\
Fundamental incompleto ou equivalente & 5.301 & 8.316 \\
Fundamental completo ou equivalente & 2.491 & 824 \\
Médio incompleto ou equivalente & 4.625 & 2.937 \\
Médio completo ou equivalente & 13.513 & 15.958 \\
Superior incompleto ou equivalente & 1.827 & 1.794 \\
Superior completo & 5.387 & 3.397 \\
\end{longtable}

Tabela \ref{tab7} observa as pessoas desempregadas que deixara de
estudar, os Nem-Nem. tem-se que essa situação acontece em maior força
nas pessoas com baixa escolaridade com até ensino médio completo e
principalmente em pessoas com fundamental incompleto. Para essas faixas
houve aumento das pessoas Nem-Nem, em especial para as pessoas com
ensino fundamental incompleto que saiu de 5.301 para 8.316 pessoas no 1º
Trimestre de 2023. Destaca-se o aumento para faixa de pessoas com Ensino
Médio Completo que saiu de 13.513 para 15.958 pessoas NEM-NEM. Para os
níveis mais elevados houve redução. No geral tem-se aumento de 405
pessoas Nem-Nem na Capital do estado, em destaque para os níveis mais
baixos de escolaridade, do 4º Trimestre de 2022 para o 1º Trimestre de
2023.

Na Tabela \ref{tab8} mostra que no 1º Trimestre de 2023 ocorre redução
do Número de jovens em Aracaju, saiu de \emph{115.733} para
\emph{99.249} jovens. A análise dos motivos dessa redução não são objeto
de estudo desse trabalho.

\begin{longtable}[]{@{}
  >{\centering\arraybackslash}p{(\columnwidth - 10\tabcolsep) * \real{0.1341}}
  >{\centering\arraybackslash}p{(\columnwidth - 10\tabcolsep) * \real{0.0854}}
  >{\centering\arraybackslash}p{(\columnwidth - 10\tabcolsep) * \real{0.2195}}
  >{\centering\arraybackslash}p{(\columnwidth - 10\tabcolsep) * \real{0.2561}}
  >{\centering\arraybackslash}p{(\columnwidth - 10\tabcolsep) * \real{0.0976}}
  >{\centering\arraybackslash}p{(\columnwidth - 10\tabcolsep) * \real{0.2073}}@{}}
\caption{\label{tab8}Ocupação e Desocupação por Faixa Etária em Aracaju,
2022.4 a 2023.1}\tabularnewline
\toprule\noalign{}
\begin{minipage}[b]{\linewidth}\centering
Trimestre
\end{minipage} & \begin{minipage}[b]{\linewidth}\centering
Faixa
\end{minipage} & \begin{minipage}[b]{\linewidth}\centering
Pessoas ocupadas
\end{minipage} & \begin{minipage}[b]{\linewidth}\centering
Pessoas desocupadas
\end{minipage} & \begin{minipage}[b]{\linewidth}\centering
Total
\end{minipage} & \begin{minipage}[b]{\linewidth}\centering
Tx. Desocupação
\end{minipage} \\
\midrule\noalign{}
\endfirsthead
\toprule\noalign{}
\begin{minipage}[b]{\linewidth}\centering
Trimestre
\end{minipage} & \begin{minipage}[b]{\linewidth}\centering
Faixa
\end{minipage} & \begin{minipage}[b]{\linewidth}\centering
Pessoas ocupadas
\end{minipage} & \begin{minipage}[b]{\linewidth}\centering
Pessoas desocupadas
\end{minipage} & \begin{minipage}[b]{\linewidth}\centering
Total
\end{minipage} & \begin{minipage}[b]{\linewidth}\centering
Tx. Desocupação
\end{minipage} \\
\midrule\noalign{}
\endhead
\bottomrule\noalign{}
\endlastfoot
2022.4 & 15\_19 & 10.222 & 9.083 & 19.304 & 47,0 \\
2022.4 & 20\_24 & 32.571 & 10.620 & 43.191 & 24,6 \\
2022.4 & 25\_29 & 44.252 & 8.986 & 53.238 & 16,9 \\
2023.1 & 15\_19 & 7.938 & 7.273 & 15.211 & 47,8 \\
2023.1 & 20\_24 & 32.568 & 8.178 & 40.746 & 20,1 \\
2023.1 & 25\_29 & 38.295 & 4.998 & 43.292 & 11,5 \\
\end{longtable}

É demonstrado também a situação empregatícia dos jovens, segundo o IBGE
\citep{ibge2023} os jovens são divididos em faixa de 15 a 29 anos, nesse
sentido foi demostrado redução de jovens de 15 a 19 anos ocupados e
desocupados para variação do último trimestre de 2022 para para o 1º
Trimestre de 2023, mas embora tenha havido redução em âmbos temos que
foi remetid oem menot impacto no núemro de despcucpados, o que demostra
em certo sentido uma piora relativa no quadro econômico-social da
capital.

Analisando os jovens de 20 a 24 anos, foi observado que há também
redução no número de ocupados e desocupados, mas a redução é ainda maior
no número de desocupados, entretanto esse fato não evidência/constata
melhora no número de jovens ocuados nessa faixa etária.

Na faixa de idade de 25 a 29 anos, tambpem ocorreu reduçao n onumero de
ocupados e desocupados no 4º Trimestre de 2022 para 1º Trimestre de
2023. Esperava-se que com a redução de jovens desocupados de 20 a 24
anos teria ocorrido a mudança de faixa e ocorrido o aumento de jovens
ocupados de 25 a 29 anos. Mas esse quadro não se concretiza, ocorrendo o
inverso do esperado em que cai o número de ocupados e desocupados, e
mais fortemente no primeiro, cerca de \emph{5.957} e \emph{3.988}
jovens, respectivamente.

\hypertarget{jovens-desocupados-por-nivel-de-instruuxe7uxe3o-em-aracaju} caiu para \texttt{15\%}
de \texttt{75,5\%} para \texttt{46\%}, respectivamente.

\begin{longtable}[]{@{}
  >{\centering\arraybackslash}p{(\columnwidth - 8\tabcolsep) * \real{0.1392}}
  >{\raggedright\arraybackslash}p{(\columnwidth - 8\tabcolsep) * \real{0.5190}}
  >{\centering\arraybackslash}p{(\columnwidth - 8\tabcolsep) * \real{0.1139}}
  >{\centering\arraybackslash}p{(\columnwidth - 8\tabcolsep) * \real{0.1139}}
  >{\centering\arraybackslash}p{(\columnwidth - 8\tabcolsep) * \real{0.1139}}@{}}
\caption{\label{tab9}Jovens Desocupados por Nivel de Instrução em
Aracaju, 2022.4 a 2023.1}\tabularnewline
\toprule\noalign{}
\begin{minipage}[b]{\linewidth}\centering
Trimestre
\end{minipage} & \begin{minipage}[b]{\linewidth}\raggedright
Nivel
\end{minipage} & \begin{minipage}[b]{\linewidth}\centering
15 a 19
\end{minipage} & \begin{minipage}[b]{\linewidth}\centering
20 a 24
\end{minipage} & \begin{minipage}[b]{\linewidth}\centering
25 a 29
\end{minipage} \\
\midrule\noalign{}
\endfirsthead
\toprule\noalign{}
\begin{minipage}[b]{\linewidth}\centering
Trimestre
\end{minipage} & \begin{minipage}[b]{\linewidth}\raggedright
Nivel
\end{minipage} & \begin{minipage}[b]{\linewidth}\centering
15 a 19
\end{minipage} & \begin{minipage}[b]{\linewidth}\centering
20 a 24
\end{minipage} & \begin{minipage}[b]{\linewidth}\centering
25 a 29
\end{minipage} \\
\midrule\noalign{}
\endhead
\bottomrule\noalign{}
\endlastfoot
2022.4 & Sem instrução e menos de 1 ano de estudo & 0 & 0 & 0 \\
2022.4 & Fundamental incompleto ou equivalente & 2.389 & 1.159 & 363 \\
2022.4 & Fundamental completo ou equivalente & 1.855 & 652 & 398 \\
2022.4 & Médio incompleto ou equivalente & 2.610 & 798 & 868 \\
2022.4 & Médio completo ou equivalente & 2.229 & 3.974 & 3.360 \\
2022.4 & Superior incompleto ou equivalente & 0 & 3.225 & 2.369 \\
2022.4 & Superior completo & 0 & 812 & 1.628 \\
2023.1 & Sem instrução e menos de 1 ano de estudo & 0 & 873 & 0 \\
2023.1 & Fundamental incompleto ou equivalente & 1.083 & 1.822 & 0 \\
2023.1 & Fundamental completo ou equivalente & 885 & 0 & 0 \\
2023.1 & Médio incompleto ou equivalente & 1.375 & 583 & 391 \\
2023.1 & Médio completo ou equivalente & 2.855 & 3.359 & 2.817 \\
2023.1 & Superior incompleto ou equivalente & 1.076 & 1.391 & 864 \\
2023.1 & Superior completo & 0 & 149 & 926 \\
\end{longtable}

Há aumento de desempregados na faixa de 20 a 24 anos, com apenas o
fundamental incompleto e médio incompleto, de \texttt{10,9\%} para
\texttt{33\%} e de \texttt{24,6\%} para \texttt{40\%}, respectivamente.

Já os jovens de 25 a 29 anos apresentaram melhora em seu percential para
os com jundamental incompleto e médio incompleto, anteriormente de
\texttt{4\%} caiu para zero e de \texttt{18\%} caiu para \texttt{7,8\%}.

Sabendo que o maior volume de jovens está concentrado na faixa de 20 a
24 anos, 8.178 jovens em estado de desocupados, há enorme preocupação ao
saber que \texttt{33\%} deles não possuem o ensiono fundamental completo
e olhando para os que não possuem ensino médio completo esse percentual
sobe para \texttt{40\%} no 1º Trimestre de 2023.

\begin{longtable}[]{@{}
  >{\centering\arraybackslash}p{(\columnwidth - 8\tabcolsep) * \real{0.1392}}
  >{\raggedright\arraybackslash}p{(\columnwidth - 8\tabcolsep) * \real{0.5190}}
  >{\centering\arraybackslash}p{(\columnwidth - 8\tabcolsep) * \real{0.1139}}
  >{\centering\arraybackslash}p{(\columnwidth - 8\tabcolsep) * \real{0.1139}}
  >{\centering\arraybackslash}p{(\columnwidth - 8\tabcolsep) * \real{0.1139}}@{}}
\caption{\label{tab10}Jovens Nem-Nem por Nivel de Instrução em Aracaju,
2022.4 a 2023.1}\tabularnewline
\toprule\noalign{}
\begin{minipage}[b]{\linewidth}\centering
Trimestre
\end{minipage} & \begin{minipage}[b]{\linewidth}\raggedright
Nivel
\end{minipage} & \begin{minipage}[b]{\linewidth}\centering
15 a 19
\end{minipage} & \begin{minipage}[b]{\linewidth}\centering
20 a 24
\end{minipage} & \begin{minipage}[b]{\linewidth}\centering
25 a 29
\end{minipage} \\
\midrule\noalign{}
\endfirsthead
\toprule\noalign{}
\begin{minipage}[b]{\linewidth}\centering
Trimestre
\end{minipage} & \begin{minipage}[b]{\linewidth}\raggedright
Nivel
\end{minipage} & \begin{minipage}[b]{\linewidth}\centering
15 a 19
\end{minipage} & \begin{minipage}[b]{\linewidth}\centering
20 a 24
\end{minipage} & \begin{minipage}[b]{\linewidth}\centering
25 a 29
\end{minipage} \\
\midrule\noalign{}
\endhead
\bottomrule\noalign{}
\endlastfoot
2022.4 & Sem instrução e menos de 1 ano de estudo & 0 & 0 & 0 \\
2022.4 & Fundamental incompleto ou equivalente & 1.166 & 1.159 & 363 \\
2022.4 & Fundamental completo ou equivalente & 593 & 407 & 398 \\
2022.4 & Médio incompleto ou equivalente & 698 & 798 & 868 \\
2022.4 & Médio completo ou equivalente & 919 & 3.649 & 3.360 \\
2022.4 & Superior incompleto ou equivalente & 0 & 764 & 475 \\
2022.4 & Superior completo & 0 & 478 & 1.628 \\
2023.1 & Sem instrução e menos de 1 ano de estudo & 0 & 873 & 0 \\
2023.1 & Fundamental incompleto ou equivalente & 785 & 1.822 & 0 \\
2023.1 & Fundamental completo ou equivalente & 0 & 0 & 0 \\
2023.1 & Médio incompleto ou equivalente & 0 & 583 & 391 \\
2023.1 & Médio completo ou equivalente & 2.255 & 3.359 & 2.817 \\
2023.1 & Superior incompleto ou equivalente & 0 & 0 & 305 \\
2023.1 & Superior completo & 0 & 149 & 263 \\
\end{longtable}

Na Tabela \ref{tab10}, Jovens Nem-Nem por Nivel de Instrução, é
explicitado que par ao s jovens de 15 a 19 anos têm que no 4º Trimestre
de 2022 cerca de \texttt{12,8\%} sem o ensino fundamental completo
estavam desempregados e deixaram de estudar e no 1º Trimestre de 2023
esse percentual cai para \texttt{10,8\%}, ou seja, dos \texttt{14,9\%}
dos jovens que não trabalhavam cerca de \texttt{10,8\%} já não estudavam
mais. Já para os jovens sem ensino médio completo o percentual não
aumentou, ou seja, dos \texttt{46\%} apenas \texttt{10,8\%} (o total dos
jovens sem ensino fundamental completo) deixaram de estudar.

A situação mais preocupante ocorre com os jovens de 20 a 24 e 25 a 29
anos, em que apenas para o sem o ensino médio completo no 4º Trimestre
de 2022 que houve diferença no percentual, os demais dados foram
identicos. Isso quer dizer que todos os jovens que dessa faixa etária
que não trabalhavam deixaram de estudar. Esse fato é de maior
relevânciana faixa de 20 a 24 anos pois concentra cerca de \texttt{40\%}
dos jovens desempregados no 1º Trimestre de 2023, percentual superior
dos \texttt{37\%} do Trimestre anterior.

\hypertarget{pessoas-ocupadas-por-setor-produtivo-em-aracaju}{%
\section{Pessoas Ocupadas por Setor Produtivo em
Aracaju}\label{pessoas-ocupadas-por-setor-produtivo-em-aracaju}}

Na Tabela \ref{tab11} é demostrado os setores que mais emprega na cidade
de Aracaju, que são: Comércio, Educação, Informação e Administração
Pública. Podemos observar que entre o 4º Trimestre de 2022 e 1º
Trimestre de 2023 os setores de Educação (Educação, Saúde Humana e
Serviço Social) e Administração Pública (Administração Pública, Defesa e
e Seguridade social) sofreram redução no número de ocupados, o que
mostra redução de empregados em dois setores mais sensíveis da estrutura
organizacional e social do município. Estes englobam ou abarcam os
setores de maior precocupação e déficit social tanto em níveis
municipal, estatual e federal.

Já os setores de Comércio e informação tiveram aumento do número de
empregados relativos a esses dois trimestres, sendo que o primeiro está
muito ligado a efeitos sazonais devido as datas comemoratidas dos dois
primeiros meses do ano. O setor de comércio aumentou em \emph{966} e o
de Informação em \emph{2.034} pessoas em seus quadros nesse 1º trimestre
de 2023.

Houve redução em 6 dos 11 setores analisados com total de \emph{9.064}
pessoas ocupadas no 1º Trimestre de 2023, sendo que o setor de Educação
perdeu \emph{3.347} e de Administração \emph{3.269} postos de emprego
comparativamente ao 4º Trimestre de 2022.

\begin{longtable}[]{@{}
  >{\raggedright\arraybackslash}p{(\columnwidth - 4\tabcolsep) * \real{0.8571}}
  >{\centering\arraybackslash}p{(\columnwidth - 4\tabcolsep) * \real{0.0714}}
  >{\centering\arraybackslash}p{(\columnwidth - 4\tabcolsep) * \real{0.0714}}@{}}
\caption{\label{tab11}Pessoas Ocupadas por Setor Produtivo em Aracaju,
2022.4 a 2023.1}\tabularnewline
\toprule\noalign{}
\begin{minipage}[b]{\linewidth}\raggedright
Setor
\end{minipage} & \begin{minipage}[b]{\linewidth}\centering
2022.4
\end{minipage} & \begin{minipage}[b]{\linewidth}\centering
2023.1
\end{minipage} \\
\midrule\noalign{}
\endfirsthead
\toprule\noalign{}
\begin{minipage}[b]{\linewidth}\raggedright
Setor
\end{minipage} & \begin{minipage}[b]{\linewidth}\centering
2022.4
\end{minipage} & \begin{minipage}[b]{\linewidth}\centering
2023.1
\end{minipage} \\
\midrule\noalign{}
\endhead
\bottomrule\noalign{}
\endlastfoot
Administração pública, defesa e seguridade social & 29.577 & 26.309 \\
Agricultura, pecuária, produção florestal, pesca e aquicultura & 1.452 &
734 \\
Alojamento e alimentação & 17.015 & 16.730 \\
Comércio, reparação de veículos automotores e motocicletas & 66.814 &
67.780 \\
Construção & 15.048 & 15.376 \\
Educação, saúde humana e serviços sociais & 62.700 & 59.353 \\
Indústria geral & 21.580 & 17.773 \\
Informação, comunicação e atividades financeiras, imobiliárias,
profissionais e administrativas & 44.676 & 46.710 \\
Outros Serviços & 23.035 & 17.413 \\
Serviços domésticos & 12.283 & 14.166 \\
Transporte, armazenagem e correio & 15.976 & 18.748 \\
\end{longtable}

A Tabela \ref{tab12}, Renda Média por Setor, mostra que dentre os quatro
setores que mais emprega em Aracaju tem-se: o setor de Comércio sendo
este o que mais emprega e o que pior remunera, e demonstrou aumento da
renda do primeiro trimestre de 2023. O segundo é o setor de Educação,
que também é o segundo em termos de remuneração, com renda média
crescente e aproximadamente de R\$ 6.176,11. Já para o terceiro, setor
de Informação, houve redução em comparação ao último trimestre de 2022,
que no 1 trimestre de 2023 finalizou com renda média aproximada de R\$
3.359,32.

O quarto setor que mais emprega no município de Aracaju é o da
Administração Pública que no primeiro trimestre de 2023 aumentou a renda
média para R\$ 8.392,68, sendo este o que melhor remunera na Capital do
estado.

\begin{longtable}[]{@{}
  >{\raggedright\arraybackslash}p{(\columnwidth - 4\tabcolsep) * \real{0.8276}}
  >{\centering\arraybackslash}p{(\columnwidth - 4\tabcolsep) * \real{0.0862}}
  >{\centering\arraybackslash}p{(\columnwidth - 4\tabcolsep) * \real{0.0862}}@{}}
\caption{\label{tab12}Renda Média por Setor em Aracaju, 2022.4 a
2023.1}\tabularnewline
\toprule\noalign{}
\begin{minipage}[b]{\linewidth}\raggedright
Setor
\end{minipage} & \begin{minipage}[b]{\linewidth}\centering
2022.4
\end{minipage} & \begin{minipage}[b]{\linewidth}\centering
2023.1
\end{minipage} \\
\midrule\noalign{}
\endfirsthead
\toprule\noalign{}
\begin{minipage}[b]{\linewidth}\raggedright
Setor
\end{minipage} & \begin{minipage}[b]{\linewidth}\centering
2022.4
\end{minipage} & \begin{minipage}[b]{\linewidth}\centering
2023.1
\end{minipage} \\
\midrule\noalign{}
\endhead
\bottomrule\noalign{}
\endlastfoot
Administração pública, defesa e seguridade social~ & 8.206,51 &
8.392,68 \\
Agricultura, pecuária, produção florestal, pesca e aquicultura &
7.142,50 & 9.466,37 \\
Alojamento e alimentação~ & 1.336,94 & 1.205,37 \\
Comércio, reparação de veículos automotores e motocicletas & 1.880,50 &
2.228,80 \\
Construção & 1.733,54 & 2.613,40 \\
Educação, saúde humana e serviços sociais & 5.321,67 & 6.176,11 \\
Indústria geral & 3.179,80 & 5.111,99 \\
Informação, comunicação e atividades financeiras, imobiliárias,
profissionais e administrativas & 3.862,04 & 3.359,32 \\
Outros Serviços & 2.010,49 & 2.640,46 \\
Serviços domésticos & 1.064,14 & 941,74 \\
Transporte, armazenagem e correio~ & 1.768,35 & 2.453,41 \\
\end{longtable}

\newpage

\bibliography{mybibfile.bib}

\address{%
José Wilas Alves de Farias\\
Universidade Federal de Sergipe\\%
\\
%
%
%
\href{mailto:wilasalvesfarias89@hotmail.com}{\nolinkurl{wilasalvesfarias89@hotmail.com}}%
}
